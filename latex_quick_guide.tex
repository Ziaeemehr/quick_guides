% https://www.stat.berkeley.edu/~paciorek/computingTips/Customizing_numbering_pages.html
% Customizing the numbering for pages, figures, sections, equations, theorems, and tables 
% To number sections, pages, figures and tables nested within chapters:
\renewcommand{\thepage}{\arabic{chapter}.\arabic{page}} 
\renewcommand{\thesection}{\arabic{chapter}.\arabic{section}}  
\renewcommand{\thetable}{\arabic{chapter}.\arabic{table}}  
\renewcommand{\thefigure}{\arabic{chapter}.\arabic{figure}}


% To number supplemental material with 'S':
\renewcommand{\thepage}{S\arabic{page}} 
\renewcommand{\thesection}{S\arabic{section}}  
\renewcommand{\thetable}{S\arabic{table}}  
\renewcommand{\thefigure}{S\arabic{figure}}

% To modify the text used at the start of a caption:
\renewcommand{\figurename}{Supplemental Material, Figure} 




\documentclass{article}
\usepackage{listings}
\usepackage{xcolor} % for setting colors

% set the default code style
\lstset{
	frame=tb, % draw a frame at the top and bottom of the code block
	tabsize=4, % tab space width
	showstringspaces=false, % don't mark spaces in strings
	numbers=left, % display line numbers on the left
	commentstyle=\color{green}, % comment color
	keywordstyle=\color{blue}, % keyword color
	stringstyle=\color{red} % string color
}


%Displaying code in LaTeX documents -----------------------------------------------------
\documentclass{article}
\usepackage{listings}
\usepackage{xcolor} % for setting colors

% set the default code style
\lstset{
	frame=tb, % draw a frame at the top and bottom of the code block
	tabsize=4, % tab space width
	showstringspaces=false, % don't mark spaces in strings
	numbers=left, % display line numbers on the left
	commentstyle=\color{green}, % comment color
	keywordstyle=\color{blue}, % keyword color
	stringstyle=\color{red} % string color
}
\begin{document}
	
	\begin{lstlisting}[language=C++, caption={C++ code using listings}]
	#include <iostream>
	int main()
	{
		// print hello to the console
		std::cout << "Hello, world!" << std::endl;
		return 0;
	}
	\end{lstlisting}
	
	\begin{lstlisting}[language=Java, caption={Java code using listings}]
	public class Hello
	{
		public static void main(String[] args)
	{
		// print hello to the console
		System.out.println("Hello, world!");
	}
	}
	\end{lstlisting}
	
\end{document}




https://tex.stackexchange.com/questions/317652/how-can-i-center-equations



% Make \cite{my reference} show name and year
% https://tex.stackexchange.com/questions/135649/make-citemy-reference-show-name-and-year
\documentclass{article}
\usepackage[round]{natbib}   % omit 'round' option if you prefer square brackets
\bibliographystyle{plainnat}
\begin{document}
\citet{Franklin1999}
\bibliography{my_bibtex}
\end{document}


\usepackage{natbib}
\citet     #textual citations, print the abbreviated author list
\citet*    #textual citations, print the full author list

\citep     #parenthetical citations, print the abbreviated author list
\citep*    #parenthetical citations, print the full author list

\citealt    #the same as \citet but without any parentheses.
\citealp    #the same as \citep but without any parentheses. 


\citeauthor{ale91}         #Alex et al.
\citeauthor*{ale91}        #Alex, Mathew, and Ravi

\citeyear{ale91}           #1991 
\citeyearpar{ale91}        #(1991)
